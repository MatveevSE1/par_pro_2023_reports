\documentclass[14pt, a4paper]{extarticle}
\usepackage[utf8]{inputenc}
\usepackage[T2A]{fontenc}
\usepackage[a4paper,
left=2cm,
right=2cm,
top=2cm,
bottom=2cm]{geometry}
\usepackage{color}
\usepackage{mathtools}
\usepackage{graphicx}
\usepackage{tocloft}
\usepackage{indentfirst}
\usepackage{enumitem}
\usepackage[russian]{babel}
\usepackage{hyperref}


\setlength{\parindent}{0.8cm}
\setlength{\parskip}{0.4cm}
\renewcommand{\contentsname}{}
\renewcommand{\cftsecleader}{\cftdotfill{\cftdotsep}}
\title{}
\author{}
\date{}
\begin{document}
\begin{titlepage}
\begin{center}
{\bfseries МИНИСТЕРСТВО НАУКИ И ВЫСШЕГО ОБРАЗОВАНИЯ \\
РОССИЙСКОЙ ФЕДЕРАЦИИ}
\\
Федеральное государственное автономное образовательное учреждение высшего образования
\\
{\bfseries «Национальный исследовательский Нижегородский государственный университет им. Н.И. Лобачевского»\\(ННГУ)
\\Институт информационных технологий, математики и механики} \\
\end{center}

\vspace{8em}

\begin{center}
Отчёт по лабораторной работе \\
«Параллельная реализация вычисления многомерных интегралов с использованием многошаговой схемы (метод трапеций).»
\end{center}

\vspace{5em}


\begin{flushright}
{\bfseries Выполнил:} студент группы\\382006-2\\Колесников Д.Е. \underline{\hspace{3cm}} \linebreak\linebreak\linebreak
{\bfseries Проверил:} младший научный\\сотрудник\\Нестеров А.Ю. \underline{\hspace{3cm}} 
\end{flushright}


\vspace{\fill}

\begin{center}
Нижний Новгород\\2023
\end{center}

\end{titlepage}

\tableofcontents
\thispagestyle{empty}
\newpage


  \pagestyle{plain}
  \setcounter{page}{3}
  
\section{Введение}

Лабораторная работа посвящена исследованию применения различных методов параллельных вычислений для решения задач, связанных с вычислением многомерных интегралов методом трапеций. 

Метод трапеций является широко используемым алгоритмом для численного вычисления многомерных интегралов. Он основывается на аппроксимации подынтегральной функции линейными функциями (трапециями) и вычислении суммы площадей этих трапеций. Однако при работе с большими объемами данных данный метод может быть неэффективным, и для повышения производительности требуется оптимизация.

Для решения этой проблемы в работе используются параллельные вычисления. В ходе лабораторной работы будет изучена возможность и эффективность применения потоков из STL и двух библиотек параллельных вычислений - OpenMP и TBB. Данные программные средства предоставляют удобный интерфейс для распараллеливания программ, и позволяют значительно повысить производительность вычисления алгоритмов численного интегрирования. 

\section{Постановка задачи}

Разработать четыре программы на языке программирования C++ для решения задачи об эффективном вычислении многомерных интегралов методом трапеций. Необходимо написать последовательную программу (не использующую параллельные вычисления) и параллельные версии (с использованием библиотек OpenMP, TBB и C++ threads). Сравнить эффективность работы различных технологий для распараллеливания алгоритма. Требуется произвести замеры времени выполнения и получить сравнительные оценки.

\section{Описание алгоритма}

  \begin{enumerate}
    \item Задать в явном виде функцию f(x) непрерывную на учаестке \([a,b]\)
     \item Определить пределы интегрирования а, b и число участков разбиения n
     \item Разбивает отрезок \([a,b]\) на \(n\) равных  интервалов длины h. Находим шаг разбиения \(h\) из формулы  \(h= \frac{b-a}{n}\). И узлы определяем из равенства \(x_i = a + i \cdot h,\; i = 0, 1, 2, ...., n\)
    \item Вычислить приближенное значение определенного интеграла \(f\) по формуле:
    
    \(f = \sum_{i = 0}^{n}(\frac{1}{2}*(x_{i+1}-x_i)*(y_{i+1}-y_i)*(z_{i+1}-z_i)*(f({x_i, y_i, z_i}) + f({x_{i+1}, y_{i+1}, z_{i+1}})));\)
  \end{enumerate}

\newpage
\section{Описание схемы распараллеливания}

  \begin{enumerate}
    \item   Разбиение области интегрирования: Исходную область интегрирования разбивается на подобласти (трапеции), которые будут вычисляться параллельно. Разбиение выполняется равномерно в зависимости от количества потоков участвующих в алгоритме.
    \item   Назначение задач: Каждой подобласти назначается задача вычисления интеграла на этой области. Задачи распределяются между потоками для параллельного выполнения.
    \item   Вычисление локальных интегралов: Каждый вычислительный ресурс вычисляет локальный интеграл на своей подобласти, используя метод трапеций. Для каждоый подобласти выполняется последовательный алгоритм.
    \item   Суммирование локальных результатов: После завершения вычислений на всех ресурсах, локальные результаты интегралов суммируются для получения общего результата интегрирования по всей области..
  \end{enumerate}



\section{Описание программной реализации}

Параллельные версии программы реализованы по аналогичному принципу - путем распараллеливания основного цикла интегрирования на несколько подобластей. Итерации цикла равномерно распределяются между потоками с использованием функций библиотек OpenMP и TBB для параллельного выполнения циклов с редукцией. Затем происходит суммирование локальных результатов.

В случае версии программы с использованием std::thread разбиение на подобласти выполняется вручную, а затем каждому потоку назначается своя область разбиения.

\section{Результаты экспериментов}


\begin{tabular}{|c||c|c|c||c|c|} \hline

Выполнялось интегрирование всех функций из тестового набора, время рабоыт программы на каждоый из функций суммировалось. Время работы представлено в милисекундах.
Версия & 1  & 2  & 3  & Среднее & Ускорение \\ \hline \hline
 Последовательная & 467 & 462 & 457 & 462 & 1 \\ \hline
 OpenMP           & 78 & 78 & 77 & 76.6 & 6 \\ \hline
 TBB              & 85 & 89 & 110 & 94.6 & 4.8 \\ \hline
 std::thread      & 75 & 74 & 76 & 75 & 6.16 \\ \hline
\end{tabular}

\section*{Выводы из результатов}
Все технологии распараллеливания продемонстрировали значительный прирост в скорости выполнения. Наиболее эффективными с точки зрения времени выполнения на рассматриваемом примере оказались версии OpenMP и std::thread. 

Различия в технической реализации и возможностях каждой технологии позволяют выбрать наиболее подходящий вариант в зависимости от конкретной задачи. 

\section*{Заключение}

В лабораторной работе было изучено применение различных методов параллельных вычислений для решения задач, связанных с вычислением многомерных интегралов методом трапеций. Результаты проведенных экспериментов подтвердили целесообразность использования параллельной схемы для увеличения скорости выполнения алгоритма. Однако, доступный прирост в скорости работы может быть ограничен объемом доступной памяти, количеством потоков и затратами на взаимодействие между потоками.

\section*{Литература}

1) Параллельное программирование с использованием OpenMP. Сысоев А.В..

2) Инструменты параллельного программирования для систем с общей памятью. К.В. Корняков, И.Б. Мееров, А.А. Сиднев, А.В. Сысоев, А.В. Шишков

\section*{Приложение}
Реализации всех версий программы доступны на GitHub

\url{https://github.com/VoyagerHere/par_pro_2023_omp_tbb_std}



\end{document}
