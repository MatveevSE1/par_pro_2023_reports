\documentclass[14pt, a4paper]{extarticle}
\usepackage[utf8]{inputenc}
\usepackage[T2A]{fontenc}
\usepackage[a4paper,
left=2cm,
right=2cm,
top=2cm,
bottom=2cm]{geometry}
\usepackage{color}
\usepackage{mathtools}
\usepackage{listings}
\usepackage{graphicx}
\usepackage{tocloft}
\usepackage{indentfirst}
\usepackage{enumitem}
\usepackage[russian]{babel}

\setlength{\parindent}{0.8cm}
\setlength{\parskip}{0.4cm}
\renewcommand{\contentsname}{}
\renewcommand{\cftsecleader}{\cftdotfill{\cftdotsep}}
\definecolor{dkgreen}{rgb}{0,0.6,0}
\definecolor{gray}{rgb}{0.5,0.5,0.5}
\definecolor{mauve}{rgb}{0.58,0,0.82}

\lstset{frame=none,
  language=C++,
  aboveskip=3mm,
  belowskip=3mm,
  showstringspaces=false,
  columns=flexible,
  basicstyle={\small\ttfamily},
  numbers=none,
  numberstyle=\tiny\color{gray},
  keywordstyle=\color{blue},
  commentstyle=\color{dkgreen},
  stringstyle=\color{mauve},
  breaklines=true,
  breakatwhitespace=true,
  tabsize=4
}

\title{}
\author{}
\date{}

\begin{document}
  \begin{titlepage}
    \begin{center}
      {\bfseries МИНИСТЕРСТВО НАУКИ И ВЫСШЕГО ОБРАЗОВАНИЯ \\
        РОССИЙСКОЙ ФЕДЕРАЦИИ}
      \\
      Федеральное государственное автономное образовательное учреждение высшего образования
      \\
      {\bfseries «Национальный исследовательский Нижегородский государственный университет им. Н.И. Лобачевского»\\(ННГУ)
        \\Институт информационных технологий, математики и механики} \\ Кафедра Математического обеспечения и суперкомпьютерных технологий \\ Направление подготовки «Программная инженерия»
    \end{center}

    \vspace{8em}

    \begin{center}
      ОТЧЕТ \\ по дисциплине «Параллельное программирование» \\ по лабораторной работе \\
      «Вычисление многомерных интегралов с использованием многошаговой схемы (метод прямоугольников)»
    \end{center}

    \vspace{5em}


    \begin{flushright}
      {\bfseries Выполнила:} студентка группы\\382008-1\\Тарасова А.А. \linebreak\linebreak\linebreak
      {\bfseries Проверил:} младший научный\\сотрудник\\Нестеров А.Ю.  
    \end{flushright}


    \vspace{\fill}

    \begin{center}
      Нижний Новгород\\2023
    \end{center}

  \end{titlepage}

  \tableofcontents
  \thispagestyle{empty}
  \newpage

  \pagestyle{plain}
  \setcounter{page}{3}

  \section{Введение}
    Задача вычисления определенного интеграла для некоторой заданной на отрезке [a,b] функции f(x) является классической задачей математического анализа. Если для функции f(x) не существует первообразных среди элементарных функций или ее вычисление может оказаться очень трудоемким, значение интеграла вычисляется с помощью квадратурных формул, одной из которых является формула прямоугольников.

    Метод прямоугольников — метод численного интегрирования функции, заключающийся в замене подынтегральной функции на константу на каждом элементарном отрезке. Если рассмотреть график подынтегральной функции, то метод будет заключаться в приближённом вычислении площади под графиком суммированием площадей конечного числа прямоугольников, ширина которых будет определяться расстоянием между соответствующими соседними узлами интегрирования, а высота — значением подынтегральной функции в этих узлах.
    
    В рамках лабораторной работы изучаются способы параллельных вычислений многомерных интегралов методом прямоугольников. Особое внимание уделяется рассмотрению библиотек OpenMP и TBB.

  \newpage

  \section{Постановка задачи}
  Необходимо реализовать три варианта программы для вычисления трехмерных интегралов методом прямоугольников с использованием языка программирования C++.
   \begin{itemize}
\itemПоследовательная программа.
\itemПрограмма с использованием библиотеки OpenMP.
\itemПрограмма с использованием библиотеки TBB.
\end{itemize}
Для проверки корректности выполнения каждой программы использовать фреймворк Google Test.
После этого необходимо сравнить время работы каждой из реализаций и сделать вывод об их эффективности.
  \newpage

  \section{Описание алгоритма}

  \begin{enumerate}
    \item Задать функцию f(x, y, z)
      
     \item Определить пределы интегрирования $a_{1}$, $b_{1}$, $a_{2}$, $b_{2}$, $a_{3}$, $b_{3}$ и шаг интегрирования h;

     \item Вычислить значения участков разбиения $n_{1}$ = $\frac{b_{1} - a_{1}}{h}$, $n_{2}$ = $\frac{b_{2} - a_{2}}{h}$, $n_{3}$ = $\frac{b_{3} - a_{3}}{h}$;

     \item Вычислить значения x, y, z в точках $x_{i}$ = $a_{1} + i*h + \frac{h}{2}$, $y_{i}$ = $a_{2} + i*h + \frac{h}{2}$, $z_{i}$ = $a_{3} + i*h + \frac{h}{2}$;
     
    \item Вычислить приближенное значение определенного интеграла F по формуле
    $F(x, y, z)$ = $hhh\displaystyle\sum_{i=0}^{n_{1} - 1}\displaystyle\sum_{j=0}^{n_{2} - 1}\displaystyle\sum_{k=0}^{n_{3} - 1}f(x_{i}, y_{j}, z_{k})$
  \end{enumerate}
  \newpage

  \section{Описание схемы распараллеливания}
Один из наиболее простых подходов распараллеливания реализации алгоритма - геометрическая декомпозиция данных, предполагающая разделение данных на части и применение к ним одного и того же алгоритма. Один из способов геометрического разделения данных - подсчет с разделением сетки интегрирования по столбцам. Вычисление функции в одном узле сетки не зависит от соседних узлов.

Параллельные реализации могут быть сведены к следующему описанию:
  \begin{enumerate}
    \item Вычислить значения x, y, z в точках;
    \item Выполнить алгоритм метода прямоугольников для каждого значения x, y, z;
    \item Сложить результаты работы всех потоков.
  \end{enumerate}


  \newpage

  \section{Результаты экспериментов}
  Для разных функций с границами $a_{1}$ = 0, $a_{2}$ = 0, $a_{3}$ = 0, $b_{1}$ = 2, $b_{2}$ = 2, $b_{3}$ = 2 и шагом интегрирования h = 0.04 произведено 10 экспериментов. Среднее время выполнения программ, а также максимальное и минимальное время выполнения отражены в таблицах:
 
  \noindent\textbf{Среднее время работы в миллисекундах:}\\\\
    \begin{tabular}{|c | c | c | c | c | c |} 
      \hline
       & $x + y + z$ & $xyz$ & $sin(x)cos(y) + z$ & $exp^x + y^{2}z$ & $x^3 + y^2 + z$ \\
      \hline
      Seq & 35 & 40  & 68 & 36 & 30 \\ 
      \hline
      OpenMP & 38 & 41  & 53 & 50 & 28 \\ 
      \hline
      TBB & 22 & 28  & 21 & 35 & 19 \\ 
      \hline
    \end{tabular}

    \noindent\textbf{Максимальное время работы в миллисекундах:}\\\\
    \begin{tabular}{|c | c | c | c | c | c |} 
      \hline
       & $x + y + z$ & $xyz$ & $sin(x)cos(y) + z$ & $exp^x + y^{2}z$ & $x^3 + y^2 + z$ \\
      \hline
      Seq & 81 & 87  & 204 & 73 & 39 \\ 
      \hline
      OpenMP & 85 & 77  & 136 & 111 & 41 \\ 
      \hline
      TBB & 46 & 79  & 38 & 95 & 33 \\ 
      \hline
    \end{tabular}

\noindent\textbf{Минимальное время работы в миллисекундах:}\\\\
    \begin{tabular}{|c | c | c | c | c | c |} 
      \hline
       & $x + y + z$ & $xyz$ & $sin(x)cos(y) + z$ & $exp^x + y^{2}z$ & $x^3 + y^2 + z$ \\
      \hline
      Seq & 22 & 19  & 27 & 21 & 20 \\ 
      \hline
      OpenMP & 17 & 18  & 28 & 22 & 18 \\ 
      \hline
      TBB & 11 & 11  & 11 & 9 & 8 \\ 
      \hline
    \end{tabular}

  \newpage

  \section{Вывод}
Для рассматриваемого параллаельного алгоритма использование многопоточности и параллельных программ является целесообразным. Наиболее эффективной по времени выполнения является программа с использованием библиотеки TBB.

  \newpage

  \section{Заключение}
В процессе выполнения лабораторной работы были изучены и реализованы 2 параллельных версии и последовательная версия метода прямоугольников. На основе результатов экспериментов были сделаны выводы о преимуществе и недостатках каждого подхода.

Если требуется наибольшая производительность и гибкость в управлении потоками, лучше использовать библиотеку TBB в процессе написания кода. Если требуется простота использования, то стоит использовать библиотеку OpenMP. Однако, библиотека OpenMP имеет меньшую производительность, чем TBB. Если важна скорость разработки и отладки программы, то предпочтительнее использовать последовательную реализацию.
 
  \newpage

    \section{Литература}
    
    \begin{enumerate}
        \item Образовательный комплекс «Параллельные численные методы». Козинов Е. А., Мееров И. Б., Сысоев А. В. (При поддержке компании Intel)
        \item  Метод прямоугольников https://ru.wikipedia.org/wiki/Метод-прямоугольников 
    \end{enumerate}

  \newpage

  \section{Приложение}

  \subsection{Реализации метода прямоугольников}
    \subsubsection{Последовательная}
  \begin{lstlisting}

double getSequential(double a1, double b1, double a2, double b2, double a3, double b3, double h, const std::function<double(double, double, double)>& fun) {
    int i, j, k, n1, n2, n3;
    double sum = 0;
    n1 = static_cast<int>((b1 - a1) / h);
    n2 = static_cast<int>((b2 - a2) / h);
    n3 = static_cast<int>((b3 - a3) / h);
    for (i = 0; i < n1; i++)
        for (j = 0; j < n2; j++)
            for (k = 0; k < n3; k++) {
                const double x = a1 + i * h + h / 2;
                const double y = a2 + j * h + h / 2;
                const double z = a3 + k * h + h / 2;
                sum += h * h * h * fun(x, y, z);
            }
    return sum;
}

  \end{lstlisting}
  \newpage
  \subsubsection{OMP}
  \begin{lstlisting}
double getParallel(double a1, double b1, double a2, double b2, double a3, double b3, double h, const std::function<double(double, double, double)>& fun) {
    int i, j, k, n1, n2, n3;
    double sum = 0;
    n1 = static_cast<int>((b1 - a1) / h);
    n2 = static_cast<int>((b2 - a2) / h);
    n3 = static_cast<int>((b3 - a3) / h);
    #pragma omp parallel for private (x, y, z, k)/reduction(+:sum)
    for (i = 0; i < n1; i++)
        for (j = 0; j < n2; j++)
            for (k = 0; k < n3; k++) {
                const double x = a1 + i * h + h / 2;
                const double y = a2 + j * h + h / 2;
                const double z = a3 + k * h + h / 2;
                sum += h * h * h * fun(x, y, z);
            }
    return sum;
  \end{lstlisting}
  \newpage
    \subsubsection{TBB}
  \begin{lstlisting}

class Integral {
 private:
    double h;
    double a1, a2, a3;
    double n1, n2, n3;
    double res;
    const std::function<double(double, double, double)>& fun;

 public:
    explicit Integral(double h, double n1, double n2, double n3, double a1, double a2, double a3,
        const std::function<double(double, double, double)>& _fun) :
        h(h), n1(n1), n2(n2), n3(n3), a1(a1), a2(a2), a3(a3), res(0), fun(_fun) {}

    Integral(const Integral& i, tbb::split) :
        h(i.h), n1(i.n1), n2(i.n2), n3(i.n3), a1(i.a1), a2(i.a2), a3(i.a3), res(0), fun(i.fun) {}

    void operator()(const tbb::blocked_range3d<int>& r) {
        int i = r.pages().begin(), i_end = r.pages().end();
        int j = r.rows().begin(), j_end = r.rows().end();
        int k = r.cols().begin(), k_end = r.cols().end();

        for (i = r.pages().begin(); i < i_end; i++)
            for (j = r.rows().begin(); j < j_end; j++)
                for (k = r.cols().begin(); k < k_end; k++) {
                    const double x = a1 + i * h + h / 2;
                    const double y = a2 + j * h + h / 2;
                    const double z = a3 + k * h + h / 2;
                    res += h * h * h * fun(x, y, z);
                }
    }

    void join(const Integral& i) {
        res += i.res;
    }
    double result() {
        return res;
    }
};

double getParallel(double a1, double b1, double a2, double b2, double a3,
    double b3, double h, const std::function<double(double, double, double)>& fun) {
    int n1 = static_cast<int>((b1 - a1) / h);
    int n2 = static_cast<int>((b2 - a2) / h);
    int n3 = static_cast<int>((b3 - a3) / h);
    Integral integral(h, n1, n2, n3, a1, a2, a3, fun);
    tbb::parallel_reduce(tbb::blocked_range3d<int>(0, n1, 0, n2, 0, n3), integral);
    return integral.result();
}

  \end{lstlisting}
  \newpage
\subsection{Тестирование}
\begin{lstlisting}

double f1(double x, double y, double z) {
    return x + y + z;
}

double f2(double x, double y, double z) {
    return x * y * z;
}

double f3(double x, double y, double z) {
    return sin(x) * cos(y) + z;
}

double f4(double x, double y, double z) {
    return exp(x) + y * y * z;
}

double f5(double x, double y, double z) {
    return x * x * x + y * y + z;
}

TEST(Parallel, Test_inegral_1) {
    ASSERT_DOUBLE_EQ(getSequential(0.0, 0.0, 0.0,
        2.0, 2.0, 2.0, 1.0, f1), getParallel(0.0, 0.0, 0.0,
        2.0, 2.0, 2.0, 1.0, f1));
}

TEST(Parallel, Test_inegral_2) {
    ASSERT_DOUBLE_EQ(getSequential(0.0, 1.0, 0.0,
        2.0, 4.0, 3.0, 1.0, f2), getParallel(0.0, 1.0, 0.0,
        2.0, 4.0, 3.0, 1.0, f2));
}

TEST(Parallel, Test_inegral_3) {
    ASSERT_DOUBLE_EQ(getSequential(4.0, 2.0, 1.0,
        6.0, 3.0, 3.0, 1.0, f3), getParallel(4.0, 2.0, 1.0,
        6.0, 3.0, 3.0, 1.0, f3));
}

TEST(Parallel, Test_inegral_4) {
    ASSERT_DOUBLE_EQ(getSequential(0.0, 1.0, 2.0, 2.0,
        3.0, 4.0, 1.0, f4), getParallel(0.0, 1.0, 2.0, 2.0,
        3.0, 4.0, 1.0, f4));
}

TEST(Parallel, Test_inegral_5) {
    ASSERT_DOUBLE_EQ(getSequential(5.0, 3.0, 7.0,
        6.0, 7.0, 8.0, 1.0, f5), getParallel(5.0, 3.0, 7.0,
        6.0, 7.0, 8.0, 1.0, f5));
}

int main(int argc, char** argv) {
    ::testing::InitGoogleTest(&argc, argv);
    return RUN_ALL_TESTS();
}

\end{lstlisting}
\end{document}