\documentclass[14pt, russian]{extarticle}
\usepackage[utf8]{inputenc}
\usepackage[T2A]{fontenc}
\usepackage{babel}
\usepackage[a4paper,
left=2cm,
right=2cm,
top=2cm,
bottom=2cm]{geometry}
\usepackage{color}
\usepackage{mathtools}
\usepackage{listings}
\usepackage{graphicx}
\usepackage{tocloft}
\usepackage{indentfirst}
\usepackage{enumitem}

\setlength{\parindent}{0.8cm}
\setlength{\parskip}{0.4cm}
\renewcommand{\contentsname}{}
\renewcommand{\cftsecleader}{\cftdotfill{\cftdotsep}}
\definecolor{dkgreen}{rgb}{0,0.6,0}
\definecolor{gray}{rgb}{0.5,0.5,0.5}
\definecolor{mauve}{rgb}{0.58,0,0.82}

\lstset{frame=none,
	language=C++,
	aboveskip=3mm,
	belowskip=3mm,
	showstringspaces=false,
	columns=flexible,
	basicstyle={\small\ttfamily},
	numbers=none,
	numberstyle=\tiny\color{gray},
	keywordstyle=\color{blue},
	commentstyle=\color{dkgreen},
	stringstyle=\color{mauve},
	breaklines=true,
	breakatwhitespace=true,
	tabsize=4
}

\title{}
\author{}
\date{}

\begin{document}
	\begin{titlepage}
		\begin{center}
			{\bfseries МИНИСТЕРСТВО НАУКИ И ВЫСШЕГО ОБРАЗОВАНИЯ \\
				РОССИЙСКОЙ ФЕДЕРАЦИИ}
			\\
			Федеральное государственное автономное образовательное учреждение высшего образования
			\\
			{\bfseries «Национальный исследовательский Нижегородский государственный университет им. Н.И. Лобачевского»\\(ННГУ)
				\\Институт информационных технологий, математики и механики} \\
		\end{center}
		
		\vspace{8em}
		
		\begin{center}
			ОТЧЕТ \\ по лабораторной работе \\
			«Повышение контраста полутонового изображения посредством линейной растяжки гистограммы»
		\end{center}
		
		\vspace{5em}
		
		
		\begin{flushright}
			{\bfseries Выполнил:} студент группы\\382006-1\\Мусин А.Н. \underline{\hspace{3cm}} \linebreak\linebreak\linebreak
			{\bfseries Проверил:} младший научный\\сотрудник\\Нестеров А.Ю. \underline{\hspace{3cm}} 
		\end{flushright}
		
		
		\vspace{\fill}
		
		\begin{center}
			Нижний Новгород\\2023
		\end{center}
		
	\end{titlepage}
	
	% Содержание
	\tableofcontents
\thispagestyle{empty}
\newpage

\pagestyle{plain}
\setcounter{page}{3}

	% Введение
	\section{Введение}
	Повышение контраста полутонового изображения является важной задачей в области обработки изображений. Одним из эффективных методов достижения этой цели является линейная растяжка гистограммы. Гистограмма представляет собой графическое отображение распределения яркости пикселей на изображении. Часто возникает ситуация, когда гистограмма полутонового изображения имеет узкое или смещенное распределение, что приводит к низкому контрасту и утрате деталей.
	
	Линейная растяжка гистограммы – это метод, позволяющий изменить диапазон яркости пикселей на изображении, чтобы улучшить его контрастность. Принцип этого метода заключается в растяжении гистограммы таким образом, чтобы минимальное значение яркости соответствовало черному цвету (0), а максимальное значение – белому цвету (255) на полутоновом изображении. При этом промежуточные значения яркости масштабируются линейно между этими крайними значениями.
	
	Цель данного отчета состоит в изучении и реализации метода линейной растяжки гистограммы для повышения контраста полутонового изображения. Будут рассмотрены основные принципы этого метода, его преимущества и ограничения. После реализации алгоритма будет проведена его апробация на различных тестовых изображениях, а также сравнение результатов с другими методами повышения контраста.
	
	Исследование и применение метода линейной растяжки гистограммы имеет практическую значимость в различных областях, таких как фотография, медицинская диагностика, видеообработка и другие. Этот метод позволяет улучшить визуальное восприятие изображений и сделать их более информативными и выразительными. 
	\newpage

	\section{Постановка задачи}
	Необходимо реализовать 3 варианта программы для изменения контраста изображений на языке C++: последовательный вариант; с использованием библиотеки OpenMP; c использованием библиотеки TBB.
	
	Результатом работы в каждом из вариантов должно быть изображение с минимальной яркостью пикселя 0 и максимальной 255.
	
	Реализации с использованием OpenMP и TBB должны корректно выполняться на произвольном числе потоков. Для проверки корректности работы программ следует использовать фреймворк для автоматического тестирования Google Test. С его помощью требуется написать не менее 5 автоматических тестов.
	
	В завершение необходимо провести сравнение времени работы 3-х вышеуказанных реализаций алгоритма и сделать общий вывод об эффективности использования параллельных схем.
	
	\newpage

	\section{Описание алгоритма}
		Алгоритм линейной растяжки гистограммы предназначен для повышения контраста полутонового изображения путем растяжения диапазона яркости. Он основан на анализе гистограммы изображения, которая представляет собой графическое отображение распределения яркости пикселей.
	
	Шаги алгоритма следующие:
	\begin{enumerate}[topsep=0pt, labelwidth=!, labelindent=0pt]
		\item Предварительная обработка: Первоначально, входное полутоновое изображение приводится к стандартному диапазону яркости, обычно от 0 до 255, чтобы обеспечить единообразность обработки.
		
		\item Вычисление гистограммы: Далее, вычисляется гистограмма изображения, то есть создается массив, в котором каждый элемент соответствует количеству пикселей с определенной яркостью. Это можно сделать путем подсчета числа пикселей для каждого значения яркости от 0 до 255.
		
		\item Растяжение гистограммы: Затем проводится растяжение гистограммы, при котором минимальное и максимальное значения яркости в изображении устанавливаются на черный и белый цвета соответственно (0 и 255). Для этого применяется линейное преобразование к каждому пикселю на основе его исходной яркости и новых крайних значений яркости.
		
		\item Пересчет значений пикселей: Каждое значение пикселя пересчитывается с использованием формулы:
		новое значение = (старое значение - минимальное значение) * (новый максимальный диапазон) / (максимальное значение - минимальное значение).
		Где старое значение - исходное значение пикселя, минимальное значение и максимальное значение - соответственно минимальное и максимальное значения яркости в изображении, новый максимальный диапазон - 255.
	\end{enumerate}
	
	Визуализация результата: После пересчета значений пикселей получается новое изображение с улучшенным контрастом. Это изображение может быть сохранено или отображено пользователю.
	
	Преимущества алгоритма линейной растяжки гистограммы включают простоту реализации и эффективность в улучшении контраста изображений с узкими или смещенными гистограммами. Однако, следует отметить, что данный метод может привести к увеличению шума на изображении, особенно если шум присутствует в темных областях изображения.
	\newpage
	
	\section{Описание схемы распараллеливания}
	Алгоритм линейной растяжки гистограммы можно эффективно распараллелить для ускорения его выполнения. Параллельное выполнение алгоритма позволяет использовать множество вычислительных ресурсов одновременно для обработки различных частей изображения.
	
	Одним из подходов к распараллеливанию алгоритма является разделение изображения на несколько равных или почти равных частей, называемых блоками. Каждый блок может быть обработан независимо друг от друга. Каждый блок будет содержать свою часть пикселей и, следовательно, свою гистограмму. Растяжение гистограммы может быть выполнено независимо для каждого блока параллельно.
	
	При распараллеливании алгоритма линейной растяжки гистограммы необходимо учесть следующие шаги:
	
	Разделение изображения: Изображение разделяется на блоки. Количество блоков может быть выбрано в зависимости от доступных вычислительных ресурсов и размера изображения.
	
	Вычисление гистограммы для каждого блока: Каждый блок обрабатывается независимо. Для каждого блока вычисляется его собственная гистограмма. Это можно сделать параллельно для всех блоков.
	
	Растяжение гистограммы для каждого блока: После вычисления гистограммы для каждого блока, растяжение гистограммы может быть применено независимо для каждого блока параллельно. Каждый пиксель в блоке может быть обработан независимо от других пикселей.
	
	Сбор результатов: После завершения растяжения гистограммы для каждого блока, результаты должны быть собраны и объединены для формирования окончательного изображения. Это может потребовать синхронизации и объединения данных из каждого блока.
	
	Распараллеливание алгоритма линейной растяжки гистограммы может значительно ускорить обработку изображений, особенно для больших изображений и систем с множеством ядер или вычислительных узлов. Однако, при выборе числа блоков и разделении изображения необходимо учитывать потребляемые ресурсы и возможное влияние на качество результата.
	\newpage
	\section{Описание OpenMP-версии программы}
	
	OpenMP-версия программы состоит из 1 файла musin\_a\_linear\_contrast\_omp.cpp.
	
	За параллелизм в отыскании выпуклых оболочек компонент связности отвечает функция:
	\begin{lstlisting}
	image linearContrastOmp(const image& img)
	\end{lstlisting}
	 принимающая на вход указатель на набор наборов точек. Параллелизм внутри данной функции реализован по компонентам связности и также обеспечивается средствами OpenMP. В ней используется pragma omp parallel for.
	\newpage
	
	\section{Описание TBB-версии программы}
	TBB-версия программы тоже состоит из 1 файла musin\_a\_linear\_contrast\_tbb.cpp.
	
	За параллелизм в отыскании выпуклых оболочек компонент связности отвечает функция:
	\begin{lstlisting}
		image linearContrastTbb(const image& img)
	\end{lstlisting}
	принимающая на вход указатель на набор наборов точек. Параллелизм внутри данной функции реализован по компонентам связности и также обеспечивается средствами TBB. В ней используется tbb::parallel\_for().
	\newpage
	
	\section{Результаты экспериментов}
	Работа программы была проверена при помощи следующих 6 автоматических тестов:
	\begin{enumerate}[topsep=0pt, labelwidth=!, labelindent=0pt]
		\item Заданное непустое изображение $10 * 10$
		\item Заданное пустое изображение $10 * 10$
		\item Случайное изображение $ 10 * 10 $
		\item Случайное изображение $ 20 * 10 $
		\item Случайное изображение $ 40 * 40 $
		\item Случайное изображение $ 1500 * 1500 $.
	\end{enumerate}
	
	\indent Все тесты завершились успешно. Время выполнения 6-го теста было измерено с помощью OpenMP-функции omp\_get\_wtime() и TBB-функции\\tick\_count::now(). Были получены следующие результаты:
	
	\begin{table}[ht]
		\centering
		\begin{tabular}{| c | c | c | c |}
			\hline
			Потоков & Время (Посл-но) & Время (OpenMP) & Время (TBB) \\ [0.5ex]
			\hline
			1 & 2.5288 & 2.6266 & 2.5802 \\
			\hline
			2 & 2.5436 & 1.7787 & 1.75611 \\
			\hline
			4 & 2.5656 & 1.3686 & 1.34427 \\
			\hline
			8 & 2.5712 & 1.2219 & 1.25046 \\
			\hline
		\end{tabular}
	\end{table}

	Рассмотренные эксперименты подтверждают корректность работы алгоритма на произвольно заданном количестве процессов.
	\newpage
	
	\section{Полученные выводы}
	Реализации алгоритма при помощи OpenMP и TBB получились достаточно схожи во времени выполнения, и одинаково быстрее последовательного варианта. На основе результатов проведённых экспериментов был получен следующий вывод: для рассмотренного параллельного алгоритма действительно целесообразно использовать многопоточность. Однако стоит заметить, что при существенном увеличении числа потоков будет происходить лишь увеличение времени работы из-за увеличения накладных расходов библиотек, обеспечивающих параллелизм, а также накладных расходов.
	\newpage
	
	\section{Заключение}
	Данная лабораторная работа позволила ознакомиться с возможностями использования параллелизма на уровне потоков, а также рассмотреть реализации OpenMP и TBB для C++.
	
	В ходе работы были успешно реализованы последовательная и 2 параллельные версии алгоритма. В ходе экспериментов удалось выяснить: на случайном изображении параллельные OpenMP и TBB-версии алгоритмов существенно быстрее, чем последовательная реализация.

	\newpage
	
	\section{Литература}
	\begin{enumerate}
		\item https://intuit.ru/studies/courses/10621/1105/lecture/17989?page=9
		\item https://habr.com/ru/articles/502986/
		\item https://sibsauktf.ru/courses/fulleren/g3.htm
	\end{enumerate}
	\newpage
	
	\section{Приложение}
	\subsection{OMP: musin\_a\_linear\_contrast\_omp.cpp}
	\begin{lstlisting}
	// Copyright 2023 Musin Alexandr
	
	#include <gtest/gtest.h>
	#include <omp.h>
	
	#include <vector>
	
	struct image {
		int leftColorLimit;
		int rightColorLimit;
		int width;
		int height;
		std::vector<std::vector<int>> data;
		explicit image(const std::vector<std::vector<int>>& vec) {
			int size = vec.size();
			if (!size) {
				width = 0;
				height = 0;
				leftColorLimit = 0;
				rightColorLimit = 0;
				return;
			}
			width = size;
			height = vec.at(0).size();
			data = vec;
			leftColorLimit = 255;
			rightColorLimit = 0;
			for (auto i : data) {
				for (auto j : i) {
					if (j > rightColorLimit) rightColorLimit = j;
					if (j < leftColorLimit) leftColorLimit = j;
				}
			}
		}
		friend bool operator==(const image& lhs, const image& rhs) {
			if (lhs.width != rhs.width || lhs.height != rhs.height) return false;
			for (int i = 0; i < lhs.width; i++) {
				for (int j = 0; j < lhs.height; j++) {
					if (lhs.data.at(i).at(j) != rhs.data.at(i).at(j)) return false;
				}
			}
			return true;
		}
	};
	
	image linearContrastSeq(const image& img) {
		image res(img);
		
		#pragma omp parallel for
		for (int i = 0; i < res.width; i++) {
			for (int j = 0; j < res.height; j++) {
				res.data.at(i).at(j) = (res.data.at(i).at(j) - res.leftColorLimit) *
				255 /
				(res.rightColorLimit - res.leftColorLimit);
			}
		}
		return res;
	}
	
	TEST(OMP, Test_5x5) {
		std::vector<std::vector<int>> img{
			std::vector<int>{183, 186, 177, 115, 193},
			std::vector<int>{135, 186, 192, 149, 121},
			std::vector<int>{162, 127, 190, 159, 163},
			std::vector<int>{126, 140, 126, 172, 136},
			std::vector<int>{111, 168, 167, 129, 182}};
		
		std::vector<std::vector<int>> resImg{
			std::vector<int>{223, 233, 205, 12, 255},
			std::vector<int>{74, 233, 251, 118, 31},
			std::vector<int>{158, 49, 245, 149, 161},
			std::vector<int>{46, 90, 46, 189, 77},
			std::vector<int>{0, 177, 174, 55, 220}};
		
		image fnRes = linearContrastSeq(image(img));
		ASSERT_EQ(image(resImg), fnRes);
	}
	
	TEST(OMP, Test_7x7) {
		std::vector<std::vector<int>> img{
			std::vector<int>{183, 186, 177, 115, 193, 135, 186},
			std::vector<int>{192, 149, 121, 162, 127, 190, 159},
			std::vector<int>{163, 126, 140, 126, 172, 136, 111},
			std::vector<int>{168, 167, 129, 182, 130, 162, 123},
			std::vector<int>{167, 135, 129, 102, 122, 158, 169},
			std::vector<int>{167, 193, 156, 111, 142, 129, 173},
			std::vector<int>{121, 119, 184, 137, 198, 124, 115}};
		
		std::vector<std::vector<int>> resImg{
			std::vector<int>{215, 223, 199, 34, 241, 87, 223},
			std::vector<int>{239, 124, 50, 159, 66, 233, 151},
			std::vector<int>{162, 63, 100, 63, 185, 90, 23},
			std::vector<int>{175, 172, 71, 212, 74, 159, 55},
			std::vector<int>{172, 87, 71, 0, 53, 148, 177},
			std::vector<int>{172, 241, 143, 23, 106, 71, 188},
			std::vector<int>{50, 45, 217, 92, 255, 58, 34}};
		
		image fnRes = linearContrastSeq(image(img));
		ASSERT_EQ(image(resImg), fnRes);
	}
	
	TEST(OMP, Test_10x10) {
		std::vector<std::vector<int>> img{
			std::vector<int>{183, 186, 177, 115, 193, 135, 186, 192, 149, 121},
			std::vector<int>{162, 127, 190, 159, 163, 126, 140, 126, 172, 136},
			std::vector<int>{111, 168, 167, 129, 182, 130, 162, 123, 167, 135},
			std::vector<int>{129, 102, 122, 158, 169, 167, 193, 156, 111, 142},
			std::vector<int>{129, 173, 121, 119, 184, 137, 198, 124, 115, 170},
			std::vector<int>{113, 126, 191, 180, 156, 173, 162, 170, 196, 181},
			std::vector<int>{105, 125, 184, 127, 136, 105, 146, 129, 113, 157},
			std::vector<int>{124, 195, 182, 145, 114, 167, 134, 164, 143, 150},
			std::vector<int>{187, 108, 176, 178, 188, 184, 103, 151, 154, 199},
			std::vector<int>{132, 160, 176, 168, 139, 112, 126, 186, 194, 139}};
		
		std::vector<std::vector<int>> resImg{
			std::vector<int>{212, 220, 197, 34, 239, 86, 220, 236, 123, 49},
			std::vector<int>{157, 65, 231, 149, 160, 63, 99, 63, 184, 89},
			std::vector<int>{23, 173, 170, 70, 210, 73, 157, 55, 170, 86},
			std::vector<int>{70, 0, 52, 147, 176, 170, 239, 141, 23, 105},
			std::vector<int>{70, 186, 49, 44, 215, 92, 252, 57, 34, 178},
			std::vector<int>{28, 63, 233, 205, 141, 186, 157, 178, 247, 207},
			std::vector<int>{7, 60, 215, 65, 89, 7, 115, 70, 28, 144},
			std::vector<int>{57, 244, 210, 113, 31, 170, 84, 162, 107, 126},
			std::vector<int>{223, 15, 194, 199, 226, 215, 2, 128, 136, 255},
			std::vector<int>{78, 152, 194, 173, 97, 26, 63, 220, 241, 97}};
		
		image fnRes = linearContrastSeq(image(img));
		ASSERT_EQ(image(resImg), fnRes);
	}
	
	TEST(OMP, Test_13x13) {
		std::vector<std::vector<int>> img{
			std::vector<int>{183, 186, 177, 115, 193, 135, 186, 192, 149, 121, 162,
				127, 190},
			std::vector<int>{159, 163, 126, 140, 126, 172, 136, 111, 168, 167, 129,
				182, 130},
			std::vector<int>{162, 123, 167, 135, 129, 102, 122, 158, 169, 167, 193,
				156, 111},
			std::vector<int>{142, 129, 173, 121, 119, 184, 137, 198, 124, 115, 170,
				113, 126},
			std::vector<int>{191, 180, 156, 173, 162, 170, 196, 181, 105, 125, 184,
				127, 136},
			std::vector<int>{105, 146, 129, 113, 157, 124, 195, 182, 145, 114, 167,
				134, 164},
			std::vector<int>{143, 150, 187, 108, 176, 178, 188, 184, 103, 151, 154,
				199, 132},
			std::vector<int>{160, 176, 168, 139, 112, 126, 186, 194, 139, 195, 170,
				134, 178},
			std::vector<int>{167, 101, 197, 102, 117, 192, 152, 156, 101, 180, 186,
				141, 165},
			std::vector<int>{189, 144, 119, 140, 129, 131, 117, 197, 171, 181, 175,
				109, 127},
			std::vector<int>{167, 156, 197, 153, 186, 165, 106, 183, 119, 124, 128,
				171, 132},
			std::vector<int>{129, 103, 119, 170, 168, 108, 115, 140, 149, 196, 123,
				118, 145},
			std::vector<int>{146, 151, 121, 155, 179, 188, 164, 128, 141, 150, 193,
				100, 134}};
		
		std::vector<std::vector<int>> resImg{
			std::vector<int>{213, 221, 198, 38, 239, 90, 221, 236, 126, 54, 159, 69,
				231},
			std::vector<int>{151, 162, 66, 103, 66, 185, 92, 28, 175, 172, 74, 211,
				77},
			std::vector<int>{159, 59, 172, 90, 74, 5, 56, 149, 177, 172, 239, 144,
				28},
			std::vector<int>{108, 74, 188, 54, 48, 216, 95, 252, 61, 38, 180, 33,
				66},
			std::vector<int>{234, 206, 144, 188, 159, 180, 247, 208, 12, 64, 216,
				69, 92},
			std::vector<int>{12, 118, 74, 33, 146, 61, 244, 211, 115, 36, 172, 87,
				164},
			std::vector<int>{110, 128, 224, 20, 195, 200, 226, 216, 7, 131, 139,
				255, 82},
			std::vector<int>{154, 195, 175, 100, 30, 66, 221, 242, 100, 244, 180,
				87, 200},
			std::vector<int>{172, 2, 249, 5, 43, 236, 133, 144, 2, 206, 221, 105,
				167},
			std::vector<int>{229, 113, 48, 103, 74, 79, 43, 249, 182, 208, 193, 23,
				69},
			std::vector<int>{172, 144, 249, 136, 221, 167, 15, 213, 48, 61, 72, 182,
				82},
			std::vector<int>{74, 7, 48, 180, 175, 20, 38, 103, 126, 247, 59, 46,
				115},
			std::vector<int>{118, 131, 54, 141, 203, 226, 164, 72, 105, 128, 239, 0,
				87}};
		
		image fnRes = linearContrastSeq(image(img));
		ASSERT_EQ(image(resImg), fnRes);
	}
	
	TEST(OMP, Test_15x15) {
		std::vector<std::vector<int>> img{
			std::vector<int>{183, 186, 177, 115, 193, 135, 186, 192, 149, 121, 162,
				127, 190, 159, 163},
			std::vector<int>{126, 140, 126, 172, 136, 111, 168, 167, 129, 182, 130,
				162, 123, 167, 135},
			std::vector<int>{129, 102, 122, 158, 169, 167, 193, 156, 111, 142, 129,
				173, 121, 119, 184},
			std::vector<int>{137, 198, 124, 115, 170, 113, 126, 191, 180, 156, 173,
				162, 170, 196, 181},
			std::vector<int>{105, 125, 184, 127, 136, 105, 146, 129, 113, 157, 124,
				195, 182, 145, 114},
			std::vector<int>{167, 134, 164, 143, 150, 187, 108, 176, 178, 188, 184,
				103, 151, 154, 199},
			std::vector<int>{132, 160, 176, 168, 139, 112, 126, 186, 194, 139, 195,
				170, 134, 178, 167},
			std::vector<int>{101, 197, 102, 117, 192, 152, 156, 101, 180, 186, 141,
				165, 189, 144, 119},
			std::vector<int>{140, 129, 131, 117, 197, 171, 181, 175, 109, 127, 167,
				156, 197, 153, 186},
			std::vector<int>{165, 106, 183, 119, 124, 128, 171, 132, 129, 103, 119,
				170, 168, 108, 115},
			std::vector<int>{140, 149, 196, 123, 118, 145, 146, 151, 121, 155, 179,
				188, 164, 128, 141},
			std::vector<int>{150, 193, 100, 134, 164, 124, 114, 187, 156, 143, 191,
				127, 165, 159, 136},
			std::vector<int>{132, 151, 137, 128, 175, 107, 174, 121, 158, 195, 129,
				137, 135, 193, 118},
			std::vector<int>{128, 143, 111, 128, 129, 176, 104, 143, 163, 113, 138,
				106, 140, 104, 118},
			std::vector<int>{128, 188, 169, 117, 117, 196, 124, 143, 170, 183, 190,
				199, 172, 125, 144}};
		
		std::vector<std::vector<int>> resImg{
			std::vector<int>{213, 221, 198, 38, 239, 90, 221, 236, 126, 54, 159, 69,
				231, 151, 162},
			std::vector<int>{66, 103, 66, 185, 92, 28, 175, 172, 74, 211, 77, 159,
				59, 172, 90},
			std::vector<int>{74, 5, 56, 149, 177, 172, 239, 144, 28, 108, 74, 188,
				54, 48, 216},
			std::vector<int>{95, 252, 61, 38, 180, 33, 66, 234, 206, 144, 188, 159,
				180, 247, 208},
			std::vector<int>{12, 64, 216, 69, 92, 12, 118, 74, 33, 146, 61, 244,
				211, 115, 36},
			std::vector<int>{172, 87, 164, 110, 128, 224, 20, 195, 200, 226, 216, 7,
				131, 139, 255},
			std::vector<int>{82, 154, 195, 175, 100, 30, 66, 221, 242, 100, 244,
				180, 87, 200, 172},
			std::vector<int>{2, 249, 5, 43, 236, 133, 144, 2, 206, 221, 105, 167,
				229, 113, 48},
			std::vector<int>{103, 74, 79, 43, 249, 182, 208, 193, 23, 69, 172, 144,
				249, 136, 221},
			std::vector<int>{167, 15, 213, 48, 61, 72, 182, 82, 74, 7, 48, 180, 175,
				20, 38},
			std::vector<int>{103, 126, 247, 59, 46, 115, 118, 131, 54, 141, 203,
				226, 164, 72, 105},
			std::vector<int>{128, 239, 0, 87, 164, 61, 36, 224, 144, 110, 234, 69,
				167, 151, 92},
			std::vector<int>{82, 131, 95, 72, 193, 18, 190, 54, 149, 244, 74, 95,
				90, 239, 46},
			std::vector<int>{72, 110, 28, 72, 74, 195, 10, 110, 162, 33, 97, 15,
				103, 10, 46},
			std::vector<int>{72, 226, 177, 43, 43, 247, 61, 110, 180, 213, 231, 255,
				185, 64, 113}};
		
		image fnRes = linearContrastSeq(image(img));
		ASSERT_EQ(image(resImg), fnRes);
	}
	\end{lstlisting}
	\newpage
		\subsection{OMP: musin\_a\_linear\_contrast\_tbb.cpp}
	\begin{lstlisting}
		// Copyright 2023 Musin Alexandr
		
		#include <gtest/gtest.h>
		#include <omp.h>
		#include <tbb/parallel_for.h>
		#include <tbb/blocked_range2d.h>
		
		#include <vector>
		
		struct image {
			int leftColorLimit;
			int rightColorLimit;
			int width;
			int height;
			std::vector<std::vector<int>> data;
			explicit image(const std::vector<std::vector<int>>& vec) {
				int size = vec.size();
				if (!size) {
					width = 0;
					height = 0;
					leftColorLimit = 0;
					rightColorLimit = 0;
					return;
				}
				width = size;
				height = vec.at(0).size();
				data = vec;
				leftColorLimit = 255;
				rightColorLimit = 0;
				for (auto i : data) {
					for (auto j : i) {
						if (j > rightColorLimit) rightColorLimit = j;
						if (j < leftColorLimit) leftColorLimit = j;
					}
				}
			}
			friend bool operator==(const image& lhs, const image& rhs) {
				if (lhs.width != rhs.width || lhs.height != rhs.height) return false;
				for (int i = 0; i < lhs.width; i++) {
					for (int j = 0; j < lhs.height; j++) {
						if (lhs.data.at(i).at(j) != rhs.data.at(i).at(j)) return false;
					}
				}
				return true;
			}
		};
		
		image linearContrastSeq(const image& img) {
			image res(img);
			tbb::parallel_for(tbb::blocked_range2d<int>(0, res.width, 0, res.height),
			[&](const tbb::blocked_range2d<int>& r) {
				for (int i = r.rows().begin(); i != r.rows().end(); ++i) {
					for (int j = r.cols().begin(); j != r.cols().end(); ++j) {
						res.data.at(i).at(j) = (res.data.at(i).at(j) - res.leftColorLimit) *
						255 /
						(res.rightColorLimit - res.leftColorLimit);
					}
				}
			});
			
			return res;
		}
		
		TEST(TBB, Test_5x5) {
			std::vector<std::vector<int>> img{
				std::vector<int>{183, 186, 177, 115, 193},
				std::vector<int>{135, 186, 192, 149, 121},
				std::vector<int>{162, 127, 190, 159, 163},
				std::vector<int>{126, 140, 126, 172, 136},
				std::vector<int>{111, 168, 167, 129, 182}};
			
			std::vector<std::vector<int>> resImg{
				std::vector<int>{223, 233, 205, 12, 255},
				std::vector<int>{74, 233, 251, 118, 31},
				std::vector<int>{158, 49, 245, 149, 161},
				std::vector<int>{46, 90, 46, 189, 77},
				std::vector<int>{0, 177, 174, 55, 220}};
			
			image fnRes = linearContrastSeq(image(img));
			ASSERT_EQ(image(resImg), fnRes);
		}
		
		TEST(TBB, Test_7x7) {
			std::vector<std::vector<int>> img{
				std::vector<int>{183, 186, 177, 115, 193, 135, 186},
				std::vector<int>{192, 149, 121, 162, 127, 190, 159},
				std::vector<int>{163, 126, 140, 126, 172, 136, 111},
				std::vector<int>{168, 167, 129, 182, 130, 162, 123},
				std::vector<int>{167, 135, 129, 102, 122, 158, 169},
				std::vector<int>{167, 193, 156, 111, 142, 129, 173},
				std::vector<int>{121, 119, 184, 137, 198, 124, 115}};
			
			std::vector<std::vector<int>> resImg{
				std::vector<int>{215, 223, 199, 34, 241, 87, 223},
				std::vector<int>{239, 124, 50, 159, 66, 233, 151},
				std::vector<int>{162, 63, 100, 63, 185, 90, 23},
				std::vector<int>{175, 172, 71, 212, 74, 159, 55},
				std::vector<int>{172, 87, 71, 0, 53, 148, 177},
				std::vector<int>{172, 241, 143, 23, 106, 71, 188},
				std::vector<int>{50, 45, 217, 92, 255, 58, 34}};
			
			image fnRes = linearContrastSeq(image(img));
			ASSERT_EQ(image(resImg), fnRes);
		}
		
		TEST(TBB, Test_10x10) {
			std::vector<std::vector<int>> img{
				std::vector<int>{183, 186, 177, 115, 193, 135, 186, 192, 149, 121},
				std::vector<int>{162, 127, 190, 159, 163, 126, 140, 126, 172, 136},
				std::vector<int>{111, 168, 167, 129, 182, 130, 162, 123, 167, 135},
				std::vector<int>{129, 102, 122, 158, 169, 167, 193, 156, 111, 142},
				std::vector<int>{129, 173, 121, 119, 184, 137, 198, 124, 115, 170},
				std::vector<int>{113, 126, 191, 180, 156, 173, 162, 170, 196, 181},
				std::vector<int>{105, 125, 184, 127, 136, 105, 146, 129, 113, 157},
				std::vector<int>{124, 195, 182, 145, 114, 167, 134, 164, 143, 150},
				std::vector<int>{187, 108, 176, 178, 188, 184, 103, 151, 154, 199},
				std::vector<int>{132, 160, 176, 168, 139, 112, 126, 186, 194, 139}};
			
			std::vector<std::vector<int>> resImg{
				std::vector<int>{212, 220, 197, 34, 239, 86, 220, 236, 123, 49},
				std::vector<int>{157, 65, 231, 149, 160, 63, 99, 63, 184, 89},
				std::vector<int>{23, 173, 170, 70, 210, 73, 157, 55, 170, 86},
				std::vector<int>{70, 0, 52, 147, 176, 170, 239, 141, 23, 105},
				std::vector<int>{70, 186, 49, 44, 215, 92, 252, 57, 34, 178},
				std::vector<int>{28, 63, 233, 205, 141, 186, 157, 178, 247, 207},
				std::vector<int>{7, 60, 215, 65, 89, 7, 115, 70, 28, 144},
				std::vector<int>{57, 244, 210, 113, 31, 170, 84, 162, 107, 126},
				std::vector<int>{223, 15, 194, 199, 226, 215, 2, 128, 136, 255},
				std::vector<int>{78, 152, 194, 173, 97, 26, 63, 220, 241, 97}};
			
			image fnRes = linearContrastSeq(image(img));
			ASSERT_EQ(image(resImg), fnRes);
		}
		
		TEST(TBB, Test_13x13) {
			std::vector<std::vector<int>> img{
				std::vector<int>{183, 186, 177, 115, 193, 135, 186, 192, 149, 121, 162,
					127, 190},
				std::vector<int>{159, 163, 126, 140, 126, 172, 136, 111, 168, 167, 129,
					182, 130},
				std::vector<int>{162, 123, 167, 135, 129, 102, 122, 158, 169, 167, 193,
					156, 111},
				std::vector<int>{142, 129, 173, 121, 119, 184, 137, 198, 124, 115, 170,
					113, 126},
				std::vector<int>{191, 180, 156, 173, 162, 170, 196, 181, 105, 125, 184,
					127, 136},
				std::vector<int>{105, 146, 129, 113, 157, 124, 195, 182, 145, 114, 167,
					134, 164},
				std::vector<int>{143, 150, 187, 108, 176, 178, 188, 184, 103, 151, 154,
					199, 132},
				std::vector<int>{160, 176, 168, 139, 112, 126, 186, 194, 139, 195, 170,
					134, 178},
				std::vector<int>{167, 101, 197, 102, 117, 192, 152, 156, 101, 180, 186,
					141, 165},
				std::vector<int>{189, 144, 119, 140, 129, 131, 117, 197, 171, 181, 175,
					109, 127},
				std::vector<int>{167, 156, 197, 153, 186, 165, 106, 183, 119, 124, 128,
					171, 132},
				std::vector<int>{129, 103, 119, 170, 168, 108, 115, 140, 149, 196, 123,
					118, 145},
				std::vector<int>{146, 151, 121, 155, 179, 188, 164, 128, 141, 150, 193,
					100, 134}};
			
			std::vector<std::vector<int>> resImg{
				std::vector<int>{213, 221, 198, 38, 239, 90, 221, 236, 126, 54, 159, 69,
					231},
				std::vector<int>{151, 162, 66, 103, 66, 185, 92, 28, 175, 172, 74, 211,
					77},
				std::vector<int>{159, 59, 172, 90, 74, 5, 56, 149, 177, 172, 239, 144,
					28},
				std::vector<int>{108, 74, 188, 54, 48, 216, 95, 252, 61, 38, 180, 33,
					66},
				std::vector<int>{234, 206, 144, 188, 159, 180, 247, 208, 12, 64, 216,
					69, 92},
				std::vector<int>{12, 118, 74, 33, 146, 61, 244, 211, 115, 36, 172, 87,
					164},
				std::vector<int>{110, 128, 224, 20, 195, 200, 226, 216, 7, 131, 139,
					255, 82},
				std::vector<int>{154, 195, 175, 100, 30, 66, 221, 242, 100, 244, 180,
					87, 200},
				std::vector<int>{172, 2, 249, 5, 43, 236, 133, 144, 2, 206, 221, 105,
					167},
				std::vector<int>{229, 113, 48, 103, 74, 79, 43, 249, 182, 208, 193, 23,
					69},
				std::vector<int>{172, 144, 249, 136, 221, 167, 15, 213, 48, 61, 72, 182,
					82},
				std::vector<int>{74, 7, 48, 180, 175, 20, 38, 103, 126, 247, 59, 46,
					115},
				std::vector<int>{118, 131, 54, 141, 203, 226, 164, 72, 105, 128, 239, 0,
					87}};
			
			image fnRes = linearContrastSeq(image(img));
			ASSERT_EQ(image(resImg), fnRes);
		}
		
		TEST(TBB, Test_15x15) {
			std::vector<std::vector<int>> img{
				std::vector<int>{183, 186, 177, 115, 193, 135, 186, 192, 149, 121, 162,
					127, 190, 159, 163},
				std::vector<int>{126, 140, 126, 172, 136, 111, 168, 167, 129, 182, 130,
					162, 123, 167, 135},
				std::vector<int>{129, 102, 122, 158, 169, 167, 193, 156, 111, 142, 129,
					173, 121, 119, 184},
				std::vector<int>{137, 198, 124, 115, 170, 113, 126, 191, 180, 156, 173,
					162, 170, 196, 181},
				std::vector<int>{105, 125, 184, 127, 136, 105, 146, 129, 113, 157, 124,
					195, 182, 145, 114},
				std::vector<int>{167, 134, 164, 143, 150, 187, 108, 176, 178, 188, 184,
					103, 151, 154, 199},
				std::vector<int>{132, 160, 176, 168, 139, 112, 126, 186, 194, 139, 195,
					170, 134, 178, 167},
				std::vector<int>{101, 197, 102, 117, 192, 152, 156, 101, 180, 186, 141,
					165, 189, 144, 119},
				std::vector<int>{140, 129, 131, 117, 197, 171, 181, 175, 109, 127, 167,
					156, 197, 153, 186},
				std::vector<int>{165, 106, 183, 119, 124, 128, 171, 132, 129, 103, 119,
					170, 168, 108, 115},
				std::vector<int>{140, 149, 196, 123, 118, 145, 146, 151, 121, 155, 179,
					188, 164, 128, 141},
				std::vector<int>{150, 193, 100, 134, 164, 124, 114, 187, 156, 143, 191,
					127, 165, 159, 136},
				std::vector<int>{132, 151, 137, 128, 175, 107, 174, 121, 158, 195, 129,
					137, 135, 193, 118},
				std::vector<int>{128, 143, 111, 128, 129, 176, 104, 143, 163, 113, 138,
					106, 140, 104, 118},
				std::vector<int>{128, 188, 169, 117, 117, 196, 124, 143, 170, 183, 190,
					199, 172, 125, 144}};
			
			std::vector<std::vector<int>> resImg{
				std::vector<int>{213, 221, 198, 38, 239, 90, 221, 236, 126, 54, 159, 69,
					231, 151, 162},
				std::vector<int>{66, 103, 66, 185, 92, 28, 175, 172, 74, 211, 77, 159,
					59, 172, 90},
				std::vector<int>{74, 5, 56, 149, 177, 172, 239, 144, 28, 108, 74, 188,
					54, 48, 216},
				std::vector<int>{95, 252, 61, 38, 180, 33, 66, 234, 206, 144, 188, 159,
					180, 247, 208},
				std::vector<int>{12, 64, 216, 69, 92, 12, 118, 74, 33, 146, 61, 244,
					211, 115, 36},
				std::vector<int>{172, 87, 164, 110, 128, 224, 20, 195, 200, 226, 216, 7,
					131, 139, 255},
				std::vector<int>{82, 154, 195, 175, 100, 30, 66, 221, 242, 100, 244,
					180, 87, 200, 172},
				std::vector<int>{2, 249, 5, 43, 236, 133, 144, 2, 206, 221, 105, 167,
					229, 113, 48},
				std::vector<int>{103, 74, 79, 43, 249, 182, 208, 193, 23, 69, 172, 144,
					249, 136, 221},
				std::vector<int>{167, 15, 213, 48, 61, 72, 182, 82, 74, 7, 48, 180, 175,
					20, 38},
				std::vector<int>{103, 126, 247, 59, 46, 115, 118, 131, 54, 141, 203,
					226, 164, 72, 105},
				std::vector<int>{128, 239, 0, 87, 164, 61, 36, 224, 144, 110, 234, 69,
					167, 151, 92},
				std::vector<int>{82, 131, 95, 72, 193, 18, 190, 54, 149, 244, 74, 95,
					90, 239, 46},
				std::vector<int>{72, 110, 28, 72, 74, 195, 10, 110, 162, 33, 97, 15,
					103, 10, 46},
				std::vector<int>{72, 226, 177, 43, 43, 247, 61, 110, 180, 213, 231, 255,
					185, 64, 113}};
			
			image fnRes = linearContrastSeq(image(img));
			ASSERT_EQ(image(resImg), fnRes);
		}
	\end{lstlisting}
	\newpage
\end{document}